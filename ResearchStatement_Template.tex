\documentclass[letterpaper, 10pt]{article}
\topmargin-2.0cm

\usepackage[colorlinks = true,
linkcolor = blue,
urlcolor  = blue,
citecolor = blue,
anchorcolor = blue]{hyperref}

\usepackage{fancyhdr}
\usepackage{pagecounting}
\usepackage[dvips]{color}

\advance\oddsidemargin-0.65in
%\advance\evensidemargin-1.5cm
\textheight9.2in
\textwidth6.75in
\newcommand\bb[1]{\mbox{\em #1}}
\def\baselinestretch{1.05}

\newcommand{\hsp}{\hspace*{\parindent}}
\definecolor{gray}{rgb}{0.4,0.4,0.4}

\usepackage{siunitx}
\sisetup{range-phrase={\text{--}},range-units=single}

\usepackage{multibib}
\newcites{main}{Bibiliography}

\begin{document}
\thispagestyle{fancy}
\lhead{}
\rhead{}
\renewcommand{\headrulewidth}{0pt}
\renewcommand{\footrulewidth}{0pt}
\fancyfoot[C]{\footnotesize \textcolor{gray}{\url{http://kratsg.github.io/cv/}}}

\pagestyle{fancy}
\lhead{\textcolor{gray}{\it Giordon Stark}}
\rhead{\textcolor{gray}{\thepage/\totalpages{}}}

\begin{center}
{\LARGE \bf RESEARCH STATEMENT}\\
\vspace*{0.1cm}
{\normalsize Giordon Stark (gstark@cern.ch)}
\end{center}

As a graduate student at the University of Chicago, in collaboration with the ATLAS experiment at the Large Hadron Collider (LHC), I feel lucky to have been involved in the continued operation of one of the largest and most complex particle physics experiments today. ATLAS~\citemain{PERF-2007-01} is one of the larger experiments at the LHC designed to detect particles beyond the Standard Model, such as supersymmetry (SUSY) and dark matter (DM), from $\SI{13}{\tera\electronvolt}$ proton-proton collisions. My interest has been in studying jets with significant substructure, and applying boosted object reconstruction techniques to searches for new physics, improving the trigger efficiency for large objects with significant substructure, studying pileup mitigation in a high luminosity environment, and answering open questions about the Standard Model.

I started with ATLAS in late 2013 getting involved in a proposal for hardware and instrumentation upgrades of the trigger system for the ATLAS detector in 2020. This was the Global Feature EXtractor (gFEX~\citemain{DPF2017gFEX}) where global meant that the full calorimeter information from the ATLAS detector is contained and analyzed on a single printed circuit board. This trigger system upgrade will allow ATLAS to improve the efficiency for objects with significant substructure greatly impacting any physics program sensitive to these objects such as my thesis analysis searching for boosted stop squarks. Even as a graduate student, I took on a significant leading role as I have (1) performed trigger-level analysis studies demonstrating the physics and benefits of this instrumentation, (2) pioneered the use of embedded operating systems on hardware within ATLAS, (3) designed a low-level networking interface for slow control and monitoring of the board, and (4) editing the technical reference manual for gFEX. I am fortunate to have joined gFEX as early as I did to be able to see it through from initial drafting stages to hopefully having it installed in ATLAS in the next few months.

At the same time I started, ATLAS was gearing up for its second run of data collection in mid-2015. The software side of it had undergone a revolution of sorts, with a new event data model (EDM) for physics analyzers. I created a physics analysis framework (xAODAnaHelpers~\citemain{giordon_stark_2015_839037}) that centralized a lot of common, repetitive work that others would do, such as calibration of jet energy or removing events that contain cosmic muons. This is one of the biggest frameworks within ATLAS and has given me the opportunity to understand a lot of the technical details of reconstruction code and computing environment. I believe the framework owes its popularity due to my focus on good, easy-to-access documentation that is automatically generated and published with every change to the framework and docker images I've developed that allows new users to start an analysis without a requiring technical expertise.

Shortly after all of this work, I joined an analysis where I lead the development in the search for new physics. This analysis is a search for SUSY involving the pair production of gluinos decaying via third-generating squarks into the lightest neutralino. Two papers have been published so far with $\SI{3.2}{\per\femto\barn}$~\citemain{SUSY-2015-10} and $\SI{36.1}{\per\femto\barn}$~\citemain{SUSY-2016-10} of data. While I have not discovered SUSY yet, I am proud that the analysis has set one of the strongest limits in both ATLAS and CMS collaborations, excluding gluino masses less than $\SI{2.0}{\tera\electronvolt}$. As part of the analysis team, my contributions have been varied including (1) developing and maintaining the software framework, (2) applying boosted object reconstruction techniques to discriminate SUSY signal against Standard Model background, (3) defining search regions to maximize discovery power, (2) studying theoretical uncertainties, and (5) performing hypothesis testing to observe a discovery.

\textbf{Frame the context of my work in terms of the institute / post-doc position I am applying for and how we match. Expect about 2 parapgraphs for this.}

I am a particle physicist that loves a highly collaborative environment. My work on searching for supersymmetry with the ATLAS detector pioneered the use of boosted object reconstruction techniques to extend the results from the Run 1 program. Jets, especially those with significant substructure, accidental or otherwise, remains a significant focus of my work. I also have a strong background in instrumentation with gFEX that will allow me to directly contribute to the maintenance, operation, upgrade, and commissioning of the detector at CERN in order to maintain a physics program for the next generation of physicists. Moving forward, I am studying jet-area based pile-up suppression techniques for forward jets in a HL-LHC environment. I would like to develop new substructure variables for reclustered jets. I am open to working on more searches for particles Beyond the Standard Model.

\vspace{0.5cm}

\begin{footnotesize}

\bibliographystylemain{atlasBibStyleWithTitle}
\bibliographymain{main}
%\nocitemain{*}

\end{footnotesize}

\end{document}
